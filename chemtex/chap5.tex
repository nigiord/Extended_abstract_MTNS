
  \documentstyle[12pt]{report}
  \nofiles                          
  \def\LATEX{\LaTeX}
  \let\TEX = \TeX               
  \setcounter{totalnumber}{5}   
  \setcounter{topnumber}{3}     
  \setcounter{bottomnumber}{3}
  \setlength{\oddsidemargin}{3.9cm}     %real measurement 1.5in
  \setlength{\textwidth}{5.7in}         %right margin is now 1in
  \setlength{\topmargin}{1cm}
  \setlength{\headheight}{.6cm}
  \setlength{\textheight}{8.5in}
  \setlength{\parindent}{1cm}
  \renewcommand{\baselinestretch}{1.5}
  \raggedbottom
  \input{init.tex}
  \input{hetisix.tex}
  \input{hetifive.tex}
  \input{furanose.tex}
  \input{pyranose.tex}
  \input{purine.tex}
  \input{six.tex}
  \input{fparts.tex}
  \input{cleft.tex}
  \input{cto.tex}
  \begin {document}    
  \setcounter{page}{35}
  \setcounter{chapter}{5}
  \textfont1=\tenrm
  \initial
  \len=4

 \centerline{CHAPTER V}
 \vspace{\len mm}
 \centerline{COMBINING STRUCTURES FROM SEVERAL MACROS}
 \vspace{\len mm}
 \centerline{1. GENERAL CONSIDERATIONS FOR COMBINING STRUCTURES}
 \centerline{IN THIS SYSTEM}
 \vspace{\len mm}
 Many individual
 structure diagrams can be typeset using just one of the macros
 together with condensed, one-line formulas; but often it will be
 necessary to combine the ring structure or the branched fragment
 from one macro with a structure part from another. To do this,
 the separate parts have to be precisely aligned horizontally and
 vertically. The typesetting of chemical equations containing
 structure diagrams also requires such alignments and is therefore
 included in this chapter.
 For the system of macros described here, alignment consists of moving
 each of the structure fragments as a whole, either within its 
 picture box or together with the box.
  
 In each structure diagram there are many different points to which
 other fragments can be attached. Similarly, one and the same   
 structure can be aligned in different ways with others to produce
 a chemical equation. For these reasons, it was not considered
 feasible to develop a symbolic language for alignment, such as
 ``attach(sixring) at(1) to(fivering) at(4).'' Instead, this
 system lets the user manipulate the placement of the structures
 at a lower level by using some of the numerical coordinates from
 the macros. While it may be considered a disadvantage that the
 user has to extract information from the macros, this method also
 puts a lot more control into the hands of the user. The type of
 user anticipated for this system will probably prefer this
 mechanism to an overdose of user-friendliness. Outright manipulation
 of coordinates is also well suited for the textual method of
 structure input employed in this system, since it helps the user
 to visualize the result.

 The information that is needed from a macro 
 to form a new structure \linebreak from fragments is the coordinate
 pair for the point of attachment in each fragment. A part of
 the code for the sixring and the relevant part of the structure
 is shown in figure 5.1 to illustrate briefly how these 
 coordinate pairs are obtained:
 If a fragment is to be attached directly to a ring position,
 for example to position 1, the coordinate pair is found in
 the unconditional part of the code as the origin of the bondline
 beginning at position 1. (The code for the bond is located by
 finding the respective line comment.) The coordinate pair in 
 this case would be (342,200). --- A fragment can also be attached
 to the end of a bond extending from the ring. These bonds are
 optional and part of the conditional code. The optional bond
 extending from position 1 is located through the comment
 ``substituent on 1.'' The x-coordinate at the end of this bond
 is $342 + 128 = 470$, since the length of the bond given in
 the code, 128 units, is the projection on the x-axis.
 The y increment from ring position 1 to the end of the bond
 is obtained from the slope of the line and the x increment
 of 128: ${\rm \Delta x(3\mbox{/}5)=77}$.
 Thus, the y-coordinate at the end of the bond is
 $200 + 77 = 277$ units. -- Appendix B lists the coordinates
 of the more commonly used points of attachment for the
 system of macros described here.
  
  \setlength{\unitlength}{.2pt}
 \begin{figure}[tb]
  \hspace{5cm}
  \begin{picture}(400,530)(0,-200)
   \put(342,200)   {\line(0,-1) {200}}
   \put(342,0)     {\line(-5,-3){171}}
   \put(171,303)   {\line(5,-3) {171}}
   \put(342,200)   {\line(5,3)  {128}}
   \thinlines
   \put(342,200)   {\vector(1,0){128}}
   \put(470,200)   {\vector(-1,0){128}}
   \put(470,200)   {\vector(0,1) {77}}
   \put(470,277)   {\vector(0,-1){77}}
   \put(320,160)   {{\scriptsize 1}}
   \put(320,0)     {{\scriptsize 2}}
   \put(370,150)   {{\scriptsize 128}}
   \put(490,220)   {{\scriptsize 77}}
  \end{picture}

 \begin{minipage}{14cm}
  \begin{verbatim}
    \begin{picture}(\pw,\pht)(-\xi,-\yi) 
     .....
     \put(342,200)     {\line(0,-1) {200}} % bond from 1 to 2
     \ifx#1Q
     \else\put(342,200){\line(5,3)  {128}} % substituent on 1
          \put(475,250){#1}            \fi   
     .....
    \end{picture}
  \end{verbatim}
 \end{minipage}

 \caption{Finding coordinates of points of attachment}
 \end{figure}   %figure 5.1 
 \setlength{\unitlength}{.1pt}

 Two conceptually different methods were used in this thesis 
 to combine structure fragments from different macros. ---     
 One method follows a suggestion in the LaTeX manual (Lamport 86,
 p. 110) to put subpictures into an encompassing picture with
 the \verb+\+put command: \\
 \centerline{$\backslash $put(x,y)\{$\backslash $begin\{picture\}
            $\ldots \backslash $end\{picture\} \ \ \}.  }
 The reference point (x,y) is the lower left corner of the 
 subpicture. When this technique is applied to the chemical
 structure macros, the macro invocation constitutes the
 subpicture. The user has to set up the encompassing picture
 and determine the coordinates of the reference points from     
 the coordinates of the points of attachment between structure
 fragments.

 The second method is somewhat less versatile; but there are
 applications for which it is preferable. In this method, the
 individual picture boxes are put next to one another on one
 line or on successive lines. The fragments in the separate
 pictures are aligned by shifting the coordinate system, i. e.
 by changing the \verb+\+xi and \verb+\+yi values in the
 picture declaration, in one or more of the pictures. 
 
 Finally, for the alignment of structures in a chemical
 equation, it is convenient to use a paragraph box
 construction (\verb+\+parbox) in addition to coordinate
 shifting. LaTeX centers a paragraph box vertically on the
 current line which contains, in the case of the chemical
 equation, textual items such as plus symbols, condensed
 formulas, and reaction arrows.

 Typical applications of all methods of combining structure 
 fragments will be described in the rest of this chapter.
 
 \pagebreak 
 \vspace{\len mm}
 \centerline{2. COMBINING FRAGMENTS TO FORM A NEW STRUCTURE}
 \vspace{\len mm}
 \noindent A. \underline{Attachment by the Subpicture Method}

 A simple example for this technique of structure-building is 
 shown in figure 5.2, where two different heterocycles are
 fitted together to produce the structure of nicotine.
 The LaTeX code to be entered by the user for this structure
 is given underneath the diagram.

 \begin{figure}[h]   % fig. 5.2
  \hspace{5cm}
  \begin{picture}(900,900)(0,0)
   \put(0,0)     {\hetisix{D}{Q}{}{Q}{Q}{Q}{D}{D}{N}  }
   \put(470,277) {\hetifive{$CH_{3}$}{Q}{Q}{Q}{Q}{S}{S}{S}{N}  }
   \put(135,330)   {A}
   \put(605,600)   {B}
  \end{picture}

 \begin{minipage}{14 cm}
  \begin{verbatim}
  \begin{picture}(900,900)(0,0)
   \put(0,0)     {\hetisix{D}{Q}{}{Q}{Q}{Q}{D}{D}{N} }
   \put(470,277) {\hetifive{$CH_{3}$}{Q}{Q}{Q}{Q}{S}{S}{S}{N}}
  \end{picture}
  \end{verbatim}
 \end{minipage}

 \caption{Nicotine structure with LaTeX code}
 \end{figure}

 The code in figure 5.2 illustrates how the user has to set up the 
 encompassing picture with the \verb+\+begin and \verb+\+end
 statements and estimated values for the picture width and height,
 both 900 units (about 3cm) in this example. Then the picture box
 of the pyridine ring is placed at the origin of the outer picture.
 Next, the points of attachment are found in the respective
 macros as ${\rm x_{AB}=470}$, ${\rm y_{AB}=277}$ for pyridine
 and ${\rm x_{BA}=0}$, ${\rm y_{BA}=0}$ for pyrrolidine.
 The coordinates of the reference point in the outer picture
 where the inner picture with the pyrrolidine ring has to be
 placed then are \\
 \centerline{${\rm x=x_{AB}-x_{BA}=470}$, \  
             ${\rm y=y_{AB}-y_{BA}=277}$.}
 It is assumed that the lower left corner of both subpictures has 
 the same coordinates, and this is the case when the macros
 are used.  When more than two ring structures are combined
 one after the other, the calculation of the reference points
 is appropriately extended.

 Since the macros for the various acyclic branched fragments also
 consist of picture boxes, these fragments can be used as    
 subpictures together with ring structures and with other
 acyclic fragments. Thus the structure of thymol in figure 5.3
 is produced from the \verb+\+sixring and the \verb+\+cdown
 macros. Again, the coordinates for the point of reference for
 the \verb+\+cdown picture are calculated from the points of
 attachment:\\
 \indent ${\rm x=x_{sixring}-x_{cdown}=\ \ 171-\ \ 33=\ \ 138}$\\
 \indent ${\rm y=y_{sixring}-y_{cdown}=-103-220=-323}$.\\
 Figure 5.4, the structure of penicillic acid, combines
 three subpictures, one from the \verb+\+cleft macro and two
 from the \verb+\+cbranch macro which draws vertical branches.

 \begin{figure}[h]   % fig. 5.3
  \hspace{6cm}
  \begin{picture}(500,1100)(0,-300)
   \put(0,0)       {\sixring{Q}{$OH$}{Q}{Q}{Q}{$CH_{3}$}{S}{S}{C} }
   \put(138,-323)  {\begin{picture}(\pw,\pht)(-\xi,-\yi)
                     \put(33,80)    {\line(0,1)  {140}}
                     \put(0,0)      {$CH$}
                     \put(0,0)      {\line(-5,-3){121}}
                     \put(80,0)     {\line(5,-3) {121}}
                     \put(-430,-150){\makebox(300,87)[r]{$H_{3}C$}}
                     \put(210,-140) {$CH_{3}$}
                    \end{picture}             }
   \end{picture}
 
  \begin{minipage}{14cm}
  \begin{verbatim}
    \begin{picture}(500,1100)(0,-300)  % estimated dimensions
      \put(0,0)     {\sixring ...                 }
      \put(138,-323){\cdown   ...                 }
    \end{picture}
   \end{verbatim}
  \end{minipage}

  \caption{Combining ring structure and acyclic subpictures}
 \end{figure}

 % I did not use the cdown macro, because this chapter needs
 % several macros and I did not want to run out of Tex memory.
 % But I tried the structure out with the macro.
 \yi=200


 \begin{figure}[t]   % fig. 5.4
  \hspace{4.5cm}
  \begin{picture}(900,600)(0,-100)
   \put(-405,160)  {\makebox(300,87)[r]{$H_{3}C$}}
   \put(0,70)      {\line(-1,1)        {100}}
   \put(-405,-185) {\makebox(300,87)[r]{$H_{2}C$}}
   \put(-9,9)      {\line(-1,-1)       {100}}
   \put(9,-9)      {\line(-1,-1)       {100}}
   \put(0,0)       {$C$}
   \put(90,33)     {\line(1,0)         {140}}
   \put(240,200)   {$O$}               
   \multiput(267,85)(26,0){2}          {\line(0,1){100}}
   \put(240,0)     {$C$}
   \put(330,33)    {\line(1,0)         {140}}
   \put(480,200)   {$OCH_{3}$}
   \put(520,85)    {\line(0,1)         {100}}
   \put(480,0)     {$C$}
   \multiput(570,20)(0,26){2}          {\line(1,0){140}}
   \put(720,0)     {$CHCOOH$}
  \end{picture}
  \caption{Combining acyclic subpictures}
 \end{figure}       
 
 % Again I did not actually use the macros here, but
 % I tried it out with them.
 
 Many structures contain condensed formula fragments between
 ring diagrams. The structure of the anesthetic piridocaine
 is shown as an example in figure 5.5.  In such a case one has
 to estimate the average horizontal space per character and move
 the second subpicture that much further to the right for each
 character, including the subscripts, in the condensed formula
 fragment. In the ten point size, in which the characters in 
 figure 5.5 are printed, the horizontal space per character
 is 6.8 points or 68 of the picture units.

  \pht=800
 \begin{figure}[h]   % fig. 5.5
  \hspace{4.5cm}
  \begin{picture}(1200,800)(0,0)
   \put(0,0)     {\sixring{$NH_{2}$}{$COOCH_{2}CH_{2}$}
                  {Q}{Q}{Q}{Q}{D}{D}{D}  }
   \put(1210,0)  {\hetisix{$H$}{Q}{Q}{Q}{Q}{}{Q}{}{$N$}  }
  \end{picture}
  \caption{Condensed formula fragment between rings}
 \end{figure}
 \reinit
 
 Other special cases occur where a diagram would become
 too crowded when the two fragments are put next to one
 another. (This does not necessarily reflect steric hindrance
 in the real, three-dimensional chemical structure.)
 In such cases the user can design a longer bondline and 
 put it into the outer picture between two points of attachment
 on macro-produced structure fragments. The structure of
 sucrose, shown in figure 5.6, illustrates this technique.
 For this structure, it was estimated that the x-offset
 between the bonding oxygen on glucose and the fructose
 ring should be at least 200 units to produce a diagram
 that does not appear crowded. Using this x-offset and
 a bonding angle of $45^{0}$ (the angle used in the pyranose
 macro for glucose), the user can then easily calculate the   
 point of reference for the fructose subpicture.


 \begin{figure}
  \hspace{3.5cm}
  \begin{picture}(1200,800)(0,-100)
   \put(0,0)     {\pyranose{$H$}{$O$}{}{$OH$}{$OH$}{}{}
                  {$HO$}{$HO$}     }
   \put(785,200) {\line(1,1){200}}
   \put(985,100) {\furanose{}{$CH_{2}OH$}{$HO$}{}{}{$OH$}{Q}{$HO$} }
  \end{picture}

  \begin{minipage}{14cm}
  \begin{verbatim}
   \put(0,0)     {\pyranose ....         }
   \put(785,200) {\line(1,1){200}}       }  % user-designed line
   \put(985,100) {\furanose ...          }
  \end{verbatim}
  \end{minipage}

  \caption{User-designed connecting bond line}
 \end{figure}

 An important special case of combining structure fragments is 
 the generation of fused ring systems. In a fused ring system
 more than one ring atom is shared between rings. ---
 A simple method for producing such diagrams is to print the
 shared bondlines from individual ring structures precisely 
 on top of each other. The bond lines have to have the same
 lengths, which is true in this system of macros for the
 five- and sixrings, the most frequently occurring ones.
 The shared lines don't appear to be heavier in the printed
 picture than other bond lines. Figure 5.7 shows the structure
 diagram of quinoline produced by this method together with
 the respective LaTeX code. These fused systems can of 
 course include substituents and multiple bond variations
 at all positions where the original macros made them
 possible. --- A relatively small number of single-ring
 fragments can produce a large number of fused systems 
 in this way, among them the very common fused systems
 of anthracene, phenanthrene, chrysene, indene, indol,
 benzimidazole, quinoline, and acridine.

 \begin{figure}    % fig. 5.7
  \hspace{6cm}
  \begin{picture}(900,900)(0,0)
   \put(0,0)   {\sixring{Q}{Q}{Q}{Q}{Q}{Q}{S}{D}{D}  }
   \put(342,0) {\hetisix{D}{Q}{Q}{Q}{Q}{Q}{D}{D}{$N$}}
  \end{picture}

  \begin{minipage}{14cm}
  \begin{verbatim}
   \begin{picture}(900,900)(0,0)
     \put(0,0)   {\sixring{Q}{Q}{Q}{Q}{Q}{Q}{S}{D}{D} }
     \put(342,0) {\hetisix{D}{Q}{Q}{Q}{Q}{Q}{D}{D}{N} }
   \end{picture}
   \end{verbatim}
   \end{minipage}
   \caption{Fusion of fully drawn rings}
 \end{figure}
  
 There are also some macros that draw fragments specifically
 designed for fusing. The following fragments \\
 \[ \fuseiv{Q}{Q}{Q}{Q}{Q}{Q}{Q}{Q}{Q} \hspace{2.6cm}
    \fuseup{Q}{Q}{Q}{Q}{Q}{Q}{Q}{Q}{Q}    \hspace{1.4cm}
    \fuseiii{Q}{Q}{Q}{Q}{Q}{Q}     \]
 are produced by the \verb+\+fuseiv, \verb+\+fuseup, and
 \verb+\+fuseiii macros. They can be attached to the five-
 and sixrings as subpictures. These fragments have the
 advantage that they can provide more options for double
 bond locations than a full ring structure within the
 constraint of nine arguments.
 
 \vspace{\len mm}
 \noindent B. \underline{Attachment by Shifting the Coordinate
                         System}

 This method is easy to use when a complex structure can be
 perceived as a series of fragments put next to one another
 horizontally, although not necessarily on exactly the same
 level. The structure of nicotine shown in figure 5.2
 belongs to this category. As an alternative to the code
 listed in figure 5.2, the following LaTeX statements can 
 be used to produce the nicotine diagram: \\
 \indent \verb+\+pw = 470 \\
 \indent \verb+\+hetisix $\ldots$ \\
 \indent \verb+\+advance \verb+\+yi by 277  \\
 \indent \verb+\+hetifive $\ldots$\ \ \ . \\
 The first statement here sets the picture width \verb+\+pw
 for the pyridine ring so that the rightside end of the
 picture box is at the x-coordinate of the point of attachment.
 Now LaTeX will put the next item on the line, in this case
 the picture box with the pyrrolidine ring, flush next to
 the pyridine box. When the structure is printed in a math
 display environment (see chapter II) where no spacing between
 items on a line is applied, there will be no space between
 the picture boxes. When the structure is put into a figure
 environment only, without math display, normal spacing
 occurs as it would happen between words on a line.
 The user then has to request negative horizontal space
 between invoking the pyridine and the pyrrolidine macro
 to correct for the spacing. A statement                
 \verb+\+hspace\{-11pt\} produced the right correction for the
 typestyle of this document.
    
 The statement \verb+\+advance \verb+\+yi by 277 causes the
 coordinate-shifting in the pyrrolidine picture.
 By increasing the y-coordinate, the pyrrolidine structure
 is shifted upwards so that the points of attachment
 of the two rings meet. In general, the coordinate shifts
 $\,\Delta $xi and $\,\Delta $yi applied to the second or any
 following picture are determined from the points of
 attachment (${\rm x_{AB}}$,${\rm y_{AB}}$) and 
 (${\rm x_{BA}}$,${\rm y_{BA}}$)
 (the terminology used for figure 5.2) as follows: \\
 \centerline{${\rm \Delta xi=x_{BA} \mbox{,}\; 
                   \Delta yi=y_{AB}-y_{BA} }$.}
 
 For vertical attachment, connecting one fragment to the 
 lower end of another by coordinate shifting, the following
 steps are necessary: The points of attachment of the
 upper and the lower fragment are shifted to the bottom
 and to the top of their respective picture boxes and
 the x-coordinates of attachment are aligned. The new
 \verb+\+xi and \verb+\+yi values are then \\
 \indent $\backslash {\rm yi_{upper}=-y_{upper} }$ \\
 \indent $\backslash {\rm xi_{lower}=x_{upper}-x_{lower} }$ \\
 \indent $\backslash {\rm yi_{lower}=\backslash pht_{lower}-
         y_{lower}+14}$(correction for vertical spacing). \\
 \indent For the structure of adenosine shown in figure 5.8
 the points of attachment on purine (at the bottom of
 N-9) and on deoxyribose (at the top of the long bond)
 have the coordinates (513,-130) and (448,380),
 respectively. Thus the structure was produced by the code 
 given underneath the diagram.
 The horizontal space is used here instead of the centering
 option. The blank lines \newpage 
 \noindent after each ring structure code are
 necessary to inform LaTeX that the next item should not
 be printed on the same line.
 
 \begin{figure}
  \hspace{5cm} \yi=130
  \purine{Q}{D}{Q}{D}{Q}{$NH_2$}{Q}{D}{Q}

  \hspace{5cm} \xi=65  \yi=534
  \furanose{N}{}{}{$OH$}{}{$OH$}{}{$HO$}

 \begin{minipage}{14cm}
 \begin{verbatim}
 \hspace{5cm}  \yi=130           \purine{ ... }
 (blank line)
 \hspace{5cm}  \xi=65   \yi=534  \furanose{ ... }
 (blank line)
 \caption{ ... }
 \end{verbatim}
 \end{minipage}

  \caption{Vertical attachment by coordinate shifting}
 \end{figure}

 
 \vspace{\len mm}
 \centerline{3. ALIGNING STRUCTURES IN AN EQUATION}
 \vspace{\len mm}
 In a chemical equation containing structure diagrams the 
 various constituents of the equation have to be horizontally
 aligned. The equation is typeset in LaTeX's horizontal mode
 on one line, the current printline. Text items such as  
 condensed formulas and plus symbols are put on the line
 as usual, their (imaginary) baseline determining the
 position of the line. The structure diagrams, as drawn
 by the macros, will not be vertically centered on the
 current line. They are drawn in picture boxes which are
 typeset on the line with the lower end of the (imaginary)
 box at the baseline of the current line. The picture boxes
 are positioned at this height without regard to the
 coordinates declared for the lower left corner of the
 box. To line up the vertical middle of the diagram in the
 box with the text of the line, one would have to shift the
 diagram downwards beyond the bottom of the declared picture.
 While this can be done, it might result in a lack of space
 under the equation, since LaTeX reserves space only according
 to the declared dimensions of the picture. Paragraph boxes
 on the other hand are normally centered on the vertical
 center of the current line. Paragraph boxes containing
 the macros are positioned somewhat differently, with 
 a point one third up from the bottom of the picture at
 the base of the current line. Thus there is one third of
 the declared picture below the base of the current line 
 which yields enough vertical space to set off the 
 equation from the succeeding text. 
 
 The equation in
 figure 5.9 was typeset by putting each macro-drawn
 diagram into a \verb+\+parbox. The y-coordinate of the
 lower end of the pictures is -300 as usual, which puts
 position 2 of the sixring and the CHOH part from the
 \verb+\+cleft macro at the base of the current line.
 The LaTeX code for the equation is shown underneath it
 in the figure.
 The y-coordinates of the structures and of the reaction
 arrow, in this case drawn by a macro, could be
 shifted individually as well to change the alignment.
 The TeX control sequence \verb+\+to ($\to $) can be used
 instead of the special reaction arrow for chemistry;
 ($\,\to $) is always centered on the line.
 
 Getting good-looking horizontal spacings within the
 equation usually requires some experimenting.
 As previously mentioned, there is no inter-item
 spacing in math mode. Therefore more explicit horizontal
 space has to be added when chemical equations are
 typeset in the math display environment.

 \begin{figure}
  \hspace{1.5cm}
  \parbox{40pt}{\sixring{Q}{$R^{2}$}{Q}{Q}{Q}{Q}{S}{S}{C} }
  \hspace{1cm} $+$ \hspace{1.5cm}
  \parbox{40pt}{\cleft{$CH_{3}$}{S}{$CHOH$}{S}                 
                      {$CH_{3}$}{Q}{}  }
  \parbox{40pt}{\cto{BF_{3}}{60^{0}}{3} }  
  \hspace{3mm}
  \parbox{40pt}{\sixring{$CH{(CH_3)}_2$}{$R^{2}$}{Q}{Q}{Q}{Q}{S}{S}{C} }
 
  \begin{minipage}{14cm}
  \begin{verbatim}
  \hspace{1.5cm}
  \parbox{40pt}{\sixring{Q}{$R^{2}$}{Q}{Q}{Q}{Q}{S}{S}{C} }
  \hspace{1cm}  $+$  \hspace{1.5cm}
  \parbox{40pt}{\cleft{$CH_{3}$}{S}{$CHOH$}{S}
                      {$CH_{3}$}{Q}{}        }
  \parbox{40pt}{\cto{BF_{3}}{60_{0}}{3}      }
  \hspace{3mm}
  \parbox{40pt}{\sixring{$CH{(CH_3)}_2$}{$R_{2}$}{Q}{Q}{Q}{Q}
                        {S}{S}{C}  }
  \end{verbatim}
  \end{minipage}
 
  \caption{Alignment in a chemical equation}
 \end{figure}
 
 \end{document}
