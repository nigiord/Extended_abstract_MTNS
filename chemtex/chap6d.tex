% \input{init.tex}
% \input{fparts.tex}
% \input{cto.tex}
% \input{bonds.tex}
  \textfont1=\tenrm
  \initial
 \len=4
 
\subsection{General Utility Macros}
\subsubsection{Macro {\tt\char"5C{}fuseiv}[9]}
  This macro typesets a fragment that is designed to be
 connected at two places to another ring system with the
 effect of fusing an additional sixring to that system.
 The fragment can be fused to positions 1 and~2 of the
 carbon fivering and the carbon sixring, and to positions
 2 and~3 of the \verb+\hetifive+ and \verb+\hetisix+ rings
 without changing
 the unitlength and the \verb+\yi+ coordinate.
 \yi=200  \pht=750
 \[ \fuseiv{$R^1$}{$R^2$}{$R^3$}{$R^4$}{D}{$R^6$}{Q}{Q}{D} \]
 
 \reinit
 \begin{description}
 \item[{\rm Arguments 1--4:}]  \rhq  All other argument
 values are used as the substituent formulas ${\rm R^1}$--${\rm R^4}$.
 \item[{\rm Argument 5:}]  An argument of ``D''
      prints a second bond between the upper point of
      attachment and position 1 (this double bond is shown
      in the diagram). \rii
 \item[{\rm Argument 6:}] \rhq  An argument of ``D'' prints
      a second bond between positions 1 and 2. All other argument
      values are used as the substituent formula ${\rm R^6}$.
 \item[{\rm Argument 7:}]  An argument of ``D''
      prints a second bond between positions 2 and 3. All other
      argument values cause no action.
 \item[{\rm Argument 8:}] \rhq  An argument of ``D'' prints
      a second bond between positions 3 and 4.  All other
      argument values are used as a second substituent in
      position 3 (not shown in the diagram).
 \item[{\rm Argument 9:}] An argument of ``D'' prints
      a second bond from the lower point of attachment to
      position 4 (this double bond is shown in the diagram).
      All other argument values cause no action.
 \end{description}
 
\subsubsection{Macro {\tt\char"5C{}fuseup}[9]}
  This macro typesets a fragment that is designed to be
 connected at two places to another ring system with the
 effect of fusing an additional sixring to that system
 at an angle. The fragment can be fused to positions
 1 and~6 of the carbon sixring and positions 3 and~4 of
 the \verb+\hetisix+ rings without changing the unitlength
 and the \verb+\yi+ coordinate.
 \advance \yi by -500
 \[ \fuseup{$R^1$}{$R^2$}{$R^3$}{$R^4$}{D}{Q}{D}{Q}{D} \]
 
 \yi=300
 \begin{description}
 \item[{\rm Arguments 1--4:}] \rhq  All other
      argument values are used as the respective substituent
      formulas ${\rm R^1}$--${\rm R^4}$.
 \item[{\rm Argument 5:}] An argument of ``D'' prints
      a second bond from the upper point of attachment to
      position 1 (the resulting double bond is shown in the
      diagram). \rii
 \item[{\rm Argument 6:}] An argument of ``D'' prints
      a second bond between positions 1 and 2. \rii
 \item[{\rm Argument 7:}] An argument of ``D'' prints
      a second bond between positions 2 and 3 (the resulting
      double bond is shown in the diagram). \rii
 \item[{\rm Argument 8:}] An argument of ``D'' prints
      a second bond between positions 3 and 4. \rii
 \item[{\rm Argument 9:}] An argument of ``D'' prints
      a second bond between position 4 and the lower point
      of attachment (the resulting double bond is shown in
      the diagram). \rii
 \end{description}
 
\subsubsection{Macro {\tt\char"5C{}fuseiii}[6]}
  This macro typesets a fragment that is designed to be
 connected at two places to another ring system with the
 effect of fusing an additional fivering to that system.
 The fragment can be fused to positions 1 and 2 of the
 carbon fivering and sixring, and to positions 2 and 3 of
 the \verb+\hetifive+ and \verb+\hetisix+ rings
 without changing the unitlength and the \verb+\yi+
 coordinate.
 \pht=600
 \[ \fuseiii{$R^1$}{$R^2$}{$R^3$}{$R^4$}{Q}{D}   \]
 
 \begin{description}
 \item[{\rm Arguments 1--4:}] \rhq  All other
      arguments are used as the respective substituent
      formulas ${\rm R^1}$--${\rm R^4}$.
 \item[{\rm Argument 5:}] \rhq  All other argument
      values are used as a second substituent in position~2
      (not shown in the diagram).
 \item[{\rm Argument 6:}] An argument of ``D''
      prints a second bond between positions 1 and 2. \rii
 \end{description}
 
\subsubsection{Macro {\tt\char"5C{}cto}[3]}
  This macro draws a reaction arrow and puts the requested
 character strings representing reagents and reaction
 conditions on top and below the arrow, respectively.
 The arrow is made long enough to accommodate the longer
 of the strings. The vertical position of the arrow can be
 changed by changing the \verb+\yi+ value.
 \pw=1500
 \[ \cto{string\  on\  top\  of\  the\  arrow}{string\  below}{26} \]
 
 \begin{description}
 \item[{\rm Arguments 1 and 2:}] The character
      strings above and below the arrow, respectively.
 \item[{\rm Argument 3:}] An integer, the number of
 characters---including subscripts---in the longer string.
 \end{description}
 
\subsubsection{Macro {\tt\char"5C{}sbond}[1]}
  This macro draws a horizontal single bond of a specified
 length, vertically centered on a line. It should be used
 for structural formulas that do not use the picture
 environment and are written on one line.
 \[ \sbond{20}  \]
 
 The argument is an integer, expressing the length of the
 bond in printer points (1pt~=~.35mm).
 
\subsubsection{Macro {\tt\char"5C{}dbond}[2]}
   This macro draws a horizontal double bond of a
 specified length. It should be used for structural
 formulas that do not use the picture environment and
 are written on one line.
 \[ \dbond{20}{10}  \]
 
 \begin{description}
 \item[{\rm Argument 1:}] An integer, expressing
      the length of the bond in printer points.
 \item[{\rm Argument 2:}] An integer, expressing
      the amount of vertical space by which the bonds have
      to be pushed together to give the desired vertical
      distance. In a document with double spacing, the
      value~19 produces appropriate spacing. The value~10 works for
      single spacing.
 \end{description}
 
\subsubsection{Macro {\tt\char"5C{}tbond}[2]}
  This macro is similar to \verb+\dbond+, except that it
 draws a triple bond:
 \[ \tbond{20}{11}  \]
 
 The meaning of the arguments is the same as in
 \verb+\dbond+. The value 20~can be used as argument~2 for double
 spacing; 11~works for single spacing.
 
\section{Common requirements for the use of the system}
 So far in this thesis it has been explained how to write
 \LaTeX\  code to produce a chemical structure diagram at a
 particular place in a document. This section will discuss the
 mandatory and the optional statements at the beginning of an
 input file that make the system of macros accessible and
 its use more practical and convenient. Figure~\ref{fg:preamble}
 contains
 these statements together with the two required declarations
 at the beginning of a \LaTeX\  file, lines (1) and~(7).
 (The line numbers are for reference only, they are not used
 in the input file.)
 
 The part of the input file preceding the \verb+\begin{document}+
 statement is called the ``preamble'' in the \LaTeX\  Manual.
 In addition to the statements shown here, the preamble usually
 contains declarations pertaining to text formatting details
 such as margin width, text height on a page, and space between
 lines.
 
 The document style option \verb+chemtex+ on line~(1) of
 Figure~\ref{fg:preamble} is necessary if the
 structure-drawing macros of this thesis are to be used for
 the preparation of a document. This statement reads the file
 \verb+chemtex.sty+ into \TeX's memory, a file that contains the
 macros, \verb+\initial+ and \verb+\reinit+ as well as those
 described above. Macro \verb+\initial+
 defines the command sequences \verb+\xi+, \verb+\yi+, \verb+\pw+,
 \verb+\pht+, \verb+\xbox+, and \verb+\len+  as integer variables
 and assigns a count register to each of them.  The use of the
 first four variables in the picture declaration and the use
 of \verb+\xbox+ in a minipage or parbox environment was
 explained in chapter II. The counter \verb+\len+ is a general
 purpose integer variable for the user. All the variables
 except \verb+\len+ are also given initial values.
 Furthermore, the unitlength for the picture environments is
 set to 0.1 printer points in \verb+\initial+. This is the
 recommended unitlength for the chemical structure diagrams,
 but it can be changed anywhere in the document.
 Line~(9) from Figure~\ref{fg:preamble}
 calls \verb+\initial+.
 
 The macro \verb+\reinit+ simply resets all the parameters
 to their initial values from \verb+\initial+. It is a
 convenience, especially for cases where more than one
 variable needs to be reset.
 
 \begin{figure}\centering
  \begin{minipage}{10cm}
  \begin{verbatim}
   (1)  \documentstyle[chemtex,...]{report}
   (2)  \setcounter{totalnumber}{4}
   (3)  \setcounter{topnumber}{2}
   (4)  \setcounter{bottomnumber}{2}
   (5)  \renewcommand{\topfraction}{.5}
   (6)  \renewcommand{\bottomfraction}{.5}
   (7)  \begin{document}
   (8) \textfont1=\tenrm
   (9) \initial
  \end{verbatim}
  \end{minipage}
  \caption{Statements at the beginning of a \LaTeX\  file}
\label{fg:preamble}
 \end{figure}
 
Note that
 other document style options (represented by the \verb+...+) can
 also be indicated on the \verb+\documentstyle+ command. See the
 \LaTeX\ manual for details.
 
 Lines (2)--(6) in Figure~\ref{fg:preamble} affect the placement of
 ``floats'' on the page. The only floats discussed in this
 thesis are the diagrams produced in the figure
 environment (see Chapter~\ref{ch:frags}). In defining the style
 of a document--the report style is designated by
 line~(1)---the \LaTeX\  program sets default values for the maximum
 total number of floats on a page (three), the maximum
 number of floats at the top of the page (two), and at
 the bottom of the page (one). These values can be
 changed for documents with an unusually large number
 of figures. Thus, lines (2)--(4) increase the
 maximum number of floats to~4, evenly distributed on
 the page. It is then necessary to change the counters
 \verb+\topfraction+ and \verb+\+bottomfraction+ to
 reflect the distribution of figures on the page.
 
 Finally, line~(9) is the optional redefinition of the
 math textfont, discussed in Chapter~\ref{ch:txltx}. This definition
 can be changed anywhere in the document.
 
